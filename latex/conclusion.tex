The simulations performed in sections \ref{sec:givenCodesSection}, \ref{sec:constantCodeRateSection} and \ref{sec:constantContraintLengthSection} show a number of features. As was expected, a decrease in code rate or an increase in constraint length in general will decrease the decoded BER. It was seen, however, that as the CER increases, eventually codes with high constraint rates will start to perform worse than codes with lower constraint rate. This means that if the channel is very poor, then increasing the constraint rate might actually have a negative effect on the decoded BER.
\\[6pt]
The code rate is seen to have a very big influence on the decoded BER in the BSC, and a smaller influence in the MBEC. This indicates that codes should be tailored to the specific channel in order to reach a target BER while maximizing throughput. If the probability of errors in the BSC is time-varying, techniques such as puncturing can be used to optimize the throughput, by increasing the code rate. To do this, feedback from the receiver about the current BER can be utilized. Then, when the BER is lower than the required BER, a puncturing matrix that gives a larger code rate can be used, increasing the throughput. The encoder and decoder must then continously coordinate how the puncturing is done to optimize the throughput with a constant BER.
\\[6pt]
In random burst error channels, such as the MBEC, as would likely be encountered in practice, it is less clear what to do. It was seen in figure \ref{fig:givenMarkovFigure}, that it is possible to achieve the same BER for two different codes, where one had code rate $1/3$ and one had code rate $1/2$. This indicates that the  throughput in burst error channels may be optimized by tailoring the constraint length to be long enough to handle the bursts in the channel. Simply increasing the constraint length is, however, not always a possibility, since the decoding complexity will increase. Therefore, other techniques such as interleaving with the depth tailored to the expected bursts in the channel could be used instead. It is hard to conclude much, except that in order to achieve optimum code rate for a given BER in such a channel, the used code must be specifically tailored to the channel.