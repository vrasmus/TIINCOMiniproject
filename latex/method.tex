As a starting point for the analysis, the following 3 sets of generator polynomials convolutional codes were given. The first subscript of each polynomial represents the code number for later reference, and the second subscript is the index of the generator polynomial associated with the code.

\begin{align*}
\begin{matrix}
\textbf{Code 1}&\textbf{Code 2}&\textbf{Code 3}\\
g_{1,1}(x) = 1 + x + x^2 + x^3 + x^6&g_{2,1}(x) = 1 + x^2 + x^3&g_{3,1}(x)=1 + x^2\\
g_{1,2}(x) = 1 + x^2 + x^3 + x^5 + x^6&g_{2,2}(x)=1 + x + x^3&g_{3,2} = 1+x+x^2\\
&g_{2,3}(x) = 1+x+x^2+x^3&g_{2,3}(x)=1+x+x^2\\
&&g_{3,4}(x) = 1+x+x^2
\end{matrix}
\end{align*}
The investigation is concerned with how the convolutional codes perform when random and burst errors are introduced in the channels. For this purpose, MATLAB scripts that would allow simulating transmission of a message over the given types of channels are created.
In each script, it is possible to adjust the input message length and the number of times the simulation should run. Increasing the number of simulation iterations will generate smoother lines of the figures, by averaging the results of multiple realizations. This reduces effects of the stochastic nature of the simulation. For the simulations presented in the paper, the result is from $25$ runs. 
An additional file, (\verb@trellisGenerator.m@), defines the convolutional code generator polynomials used. The code polynomials used for coding and decoding may be easily specified by modifying this file.
The following sections present the methods used for simulation.
\subsection{Binary Symmetric Channel}
The file \verb@randomErrors.m@ presents the method of simulation of random errors in a BSC. In the script, it is possible to modify the following simulation parameters:
\begin{itemize}\setlength\itemsep{0pt}
   \item \verb@maxCER@: Maximum Channel bit Error Rate (CER) that is simulated.
   \item \verb@CERLevels@: The number of different CER levels for the simulation. This value determines the granularity of the steps as the CER is increased from $0$ to \verb@maxCER@.
\end{itemize}
For the specified number of iterations, the script generates a message of the given length. The message is then encoded using the code definitions from \verb@trellisGenerator.m@ file and for each of the codes, the encoded data is altered using the different levels of CER. For a given CER, the exact amount of bits corresponding to that CER is altered, independently positioned. The alteration of bits at random positions in the message simulates the channel noise. For each of the error levels, the received codeword is then decoded and compared to the transmitted version, calculating the Bit Error Rate (BER).

\subsection{Deterministic Bursts}

The file \verb@burstErrors.m@ presents the method of simulation of deterministic burst errors. In the script, it is possible to modify the following simulation parameters:
\begin{itemize}\setlength\itemsep{0pt}
   \item \verb@BurstLevels@: The number of different error burst lengths that are introduced on the channel. Set to $n$, bursts of length $\{0, 1, \dots , n-1 \}$ are simulated.
   \item \verb@nBursts@: The total number of error bursts to be introduced on the channel.
   \item \verb@burstSep@: The number of unaffected bits between error bursts. Must be large enough that each burst does not influence adjacent bursts.
\end{itemize}
In a similar manner as in the BSC, for the specified number of iterations, the script generates a message of the given length. The message is then encoded using the code definitions from \verb@trellisGenerator.m@ file and for each of the codes, the encoded data is altered using by introducing errors in bursts of a specific length. The burst error represents a channel being temporarily affected by interference. For each burst length, the data is then decoded and compared to the transmitted version, calculating the number of message errors per burst of a specific length.
\subsection{Markov Burst Error Channels}
While the previous channels are easy to deal with in theory, in many cases it does not match the reality well. In practice, errors are neither entirely independant and random, nor deterministic bursts, but rather stochastic with correlation. Effects such as deep fades in wireless channels cause the probability of a symbol being in error to be much larger if the previous symbol was an error. This causes errors to appear in bursts. A simple model of such a channel can be created by using a two-state Markov chain, as shown on figure \ref{fig:markovChain}. For each time-step (each bit in the channel), the current state of the chain is updated and a bit is output. The transition probabilities $p_{01}$ and $p_{10}$ determines how the error pattern will behave. $p_{01}$ is the probability of starting a burst, and $p_{10}$ is the probability of ending the current burst. In this way, the length of bursts, BL, will be geometrically distributed:\todo{Can add a plot of this for chosen p10 if room for it.}
\begin{equation}
P(\text{BL} = k) = (1-p_{10})^{k-1}  p_{10}
\end{equation}
meaning that smaller bursts are more likely than long bursts. The Markov burst error channel will be simulated by basing the error pattern on this Markov chain.
\begin{figure}
\centering
\begin{tikzpicture}[->, >=stealth', auto, semithick, node distance=3cm]
\centering
\node[state] (A) at (0,0) {$0$};
\node[state] (D) at (3,0) {$1$};
\path(A) edge[loop left] node{$1-p_{01}$} (A);
\path(A) edge[bend left] node{$p_{01}$} (D);
\path(D) edge[bend left] node{$p_{10}$} (A);
\path(D) edge[loop right] node{$1-p_{10}$} (D);
\end{tikzpicture}
\caption{Markov Chain for Burst Channel Simulation.}
\end{figure}

The file \verb@markovErrors.m@ contains the implementation of this method for simulation of these random burst errors. It is possible to modify the following simulation parameters:
\begin{itemize}\setlength\itemsep{0pt}
\item \verb@burstProbabilityMax@, max value for $p_{01}$ to be simulated.
\item \verb@burstProbabilityLevels@, number of different levels of $p_{01}$ to be simulated. Determines the granularity of the simulation.
\item \verb@p10@, the transition probability from state $1$ to $0$
\end{itemize}
For each of the specified number of simulations, a message is generated and encoded using the code definitions from \verb@trellisGenerator.m@ and for each of these transmission through a number of Markov channels, where the transition parameter $p_{01}$ is varied according to the specified parameters, is simulated. The CER of each Markov channel is calculated. The received codeword is then decoded and compared to the real message. Finally, the BER is calculated.

\subsection{Additional Codes}
The results for the given codes are presented in section \ref{sec:givenCodesSection}. While it is possible to compare the codes, it is not entirely fair to compare codes with both varying code rate and constraint length. It is more reasonable to compare the performance of codes with the same code rate or same constraint length, to show the effect of changing only one of the parameters.
In order to do so, additional codes must be introduced for comparison. These codes are presented as Code 4 to 9, and specified by the generator polynomials below:

\begin{align*}
\begin{matrix}
\textbf{Code 4}&\textbf{Code 5}&\textbf{Code 6}\\
g_{4,1}(x) = 1 + x^2 + x^3&g_{5,1}(x) = 1 + x^2&g_{6,1}(x)=1 + x + x^2 + x^3 + x^6\\
g_{4,2}(x) = 1+x+x^2+x^3&g_{5,2}(x) = 1+x+x^2&g_{6,2}(x)=1 + x^2 + x^3 + x^5 + x^6\\
&&g_{6,3}(x)=1 + x + x^2 + x^3 + x^4 + x^5 + x^6\\
\end{matrix}\\
\begin{matrix}
\textbf{Code 7}&\textbf{Code 8}&\textbf{Code 9}\\
g_{7,1}(x) = 1 + x^2&g_{8,1}(x) = 1 + x + x^2 + x^3 + x^6&g_{9,1}(x)=1 + x^2 + x^3\\
g_{7,2}(x) = 1+x+x^2&g_{8,2}(x) = 1 + x + x^3 + x^4 + x^6&g_{9,2}(x)=1 + x + x^3\\
g_{7,3}(x) = 1+x+x^2&g_{8,3}(x) = 1 + x^2 + x^3 + x^5 + x^6&g_{9,3}(x)=1+x+x^2+x^3\\
&g_{8,4}(x) = 1 + x + x^2 + x^3 + x^4 + x^5 + x^6&g_{9,4}(x)=1+x+x^2+x^3\\
\end{matrix}
\end{align*}
These codes can be grouped with the given codes by code rate and by constraint length, as follows:
\begin{table}[h]
\centering
\begin{tabular}{ll}
\hline
Code rate &  Codes \\ \hline
1/2 & Code 1, Code 4, Code 5 \\
1/3 & Code 2, Code 6, Code 7 \\
1/4 & Code 3, Code 8, Code 9 \\
\end{tabular}
\caption{\textit{Grouping of codes based on code rate}\label{tab:codeRate}}
\end{table}

\begin{table}[h]
\centering
\begin{tabular}{ll}
\hline
Constraint length &  Codes \\ \hline
3 & Code 3, Code 5, Code 7 \\
4 & Code 2, Code 4, Code 9 \\
7 & Code 1, Code 6, Code 8 \\
\end{tabular}
\caption{\textit{Grouping of codes based on constraint length}\label{tab:constraintLength}}
\end{table}
All of these groups can be chosen for simulation in \verb@trellisGenerator.m@. The results obtained for the constant code rate of 1/2 and variable constraint length are presented in section \ref{sec:constantCodeRateSection}. Similarly, the results obtained by having a constant constraint length of 4 and variable code lengths are presented in section \ref{sec:constantContraintLengthSection}.